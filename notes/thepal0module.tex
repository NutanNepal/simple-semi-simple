\documentclass[11pt]{article}
\usepackage[margin=1in]{geometry}
\usepackage{parskip}
\usepackage{amssymb,amsmath,amsfonts,verbatim,amsthm, amscd}
\usepackage[dvipsnames]{xcolor}
\usepackage[breakable,skins]{tcolorbox}
\usepackage{graphicx,tikz-cd,adjustbox}
\usepackage{float} % breqn can sometimes conflict with amsmath, be mindful
\usepackage{scalerel}
\usepackage{stackengine,wasysym}
\usepackage{lipsum}
\usepackage[backend=biber]{biblatex} % Using biber backend is recommended

\newtcolorbox{mybox}[2][]{
    arc=0mm, enhanced, frame hidden, breakable
}
\usepackage{mathtools}

\usepackage{subcaption}% <-- added

\usepackage{array} % core package
\usepackage[sc]{mathpazo} % Palatino font with small caps (like Ravi Vakil's Rising Sea)
\usepackage[scr=euler]{mathalfa}
\usepackage[unicode=true]{hyperref} % permits navigating on the pdf
\hypersetup{pdfencoding=auto}
\usepackage{cleveref} % smart referencing
\usepackage{caption} %

% Define math commands
\newcommand{\calL}{\mathcal{L}}
\newcommand{\calH}{\mathcal{H}}
\newcommand{\bz}{\mathbb{Z}}
\newcommand{\lt}{\normalfont\text{lt}}
\newcommand{\calA}{\mathcal{A}}
\newcommand{\calS}{\mathcal{S}}
\newcommand{\ind}{\text{\normalfont Ind}}
\newcommand{\stab}{\text{\normalfont Stab}}
\newcommand{\chow}{\normalfont\text{\underline{H}}}
\newcommand{\hilbM}{\normalfont\text{\underline{H}}_\text{M}}
\newcommand{\aughilbM}{\normalfont\text{H}_\text{M}}
\newcommand{\rank}{\normalfont\text{rk}}
\newcommand{\corank}{\normalfont\text{crk}}
\newcommand{\flats}{\normalfont\text{Flats}}
\newcommand{\flags}{\normalfont\text{Flags}}
\newcommand{\matM}{\normalfont\text{M}}
\newcommand{\hilbMmodF}{\normalfont\text{\underline{H}}_\text{M/F}}
\newcommand{\M}{\normalfont\text{M}}
\newcommand{\N}{\normalfont\text{N}}
\newcommand{\e}{\normalfont\text{e}}
\newcommand{\rk}{\normalfont\text{rk}}
\newcommand{\chibar}{\overline{\chi}}
\newcommand{\E}{\normalfont\text{E}}
\newcommand{\ZZ}{\mathbb{Z}}
\newcommand{\Ebasis}[1]{\mathrm{#1}}
\newcommand{\Cbasis}[1]{\mathrm{c}^{#1}}
\newcommand{\Zbasis}[1]{\mathrm{z}^{#1}}
\newcommand{\pal}{\normalfont\text{Pal}}



\newcommand{\qbinom}[2]{\begin{bmatrix} #1 \\ #2 \end{bmatrix}_q}

% Theorem environments
\newtheorem{theorem}{Theorem}
\newtheorem{corollary}[theorem]{Corollary}
\newtheorem{lemma}[theorem]{Lemma}
\newtheorem{proposition}[theorem]{Proposition}
\newtheorem{conjecture}[theorem]{Conjecture}
\newtheorem{example}[theorem]{Example}

\theoremstyle{remark}
\newtheorem*{remark}{Remark} % Unnumbered remark
% \theoremstyle{definition} % For proof environment styling - amsthm handles this with \begin{proof}
% \newtheorem*{proof_env}{Proof}

\newcommand{\nutan}[1]{{\color{violet} \sf $\clubsuit$ Nutan: [#1]}}

\title{\texorpdfstring{Notes (Inverse Kazhdan-Lusztig Polynomials under Deletion)}{Notes (Inverse Kazhdan-Lusztig Polynomials under Deletion)}}
%\author{Nutan Nepal}
%\date{\today}

\begin{document}
{\textbf{Notes: The module \(\mathcal{S}(M)\)}}\hfill {\small{\today}}

\hrulefill % Horizontal line after title block

Let \(\M\) be a matroid with lattice of flats \(\mathcal{L}(\M)\).
The module \(\calH(\M)\) is the free \(\ZZ[x, x^{-1}]\)-module with a standard basis
indexed by the flats of \(\M\). The elements of \(\calH(\M)\) are formal sums of the form
\[
\alpha = \sum_{F \in \mathcal{L}(\M)} \alpha_F \cdot\Ebasis{F}
\]
where \(\alpha_F \in \ZZ[x, x^{-1}]\).

There is a subgroup \(\calH_p(\M)\) of \(\calH(\M)\) consisting of all
elements \(\alpha\) such that for every flat \(F \in \mathcal{L}(\M)\):
\begin{enumerate}
    \item \(\alpha_F \in \ZZ[x]\)
    \item \(\sum_{G \ge F} x^{\rank(F)-\rank(G)} \alpha_G\)
    satisfies the palindromic condition: \(f(x) = f(x^{-1})\).
\end{enumerate}

We relax the first condition and define another subgroup
\(\calH'_p(\M)\) of \(\calH(\M)\) consisting of all elements
\(\alpha\) that satisfy the second condition.

We define a new basis of
\(\calH(\M)\) given by
\[
\Cbasis{F} = \sum_{G\le F} \Cbasis{F}_G\cdot\Ebasis{G} :=
\sum_{G \le F} x^{\rank(F)-\rank(G)} \chow_{\M_G^F}(x^{-2}) \cdot \Ebasis{G}
\]
where \(\chow_{\M}\) is the Chow polynomial of the matroid \(\M\).

Let \(\pal(n)\) be the ring of Laurent polynomials that satisfy
\(f(x) = x^n f(x^{-1})\).
We note that 
\[\Cbasis{F}_F = 1 \in \pal(0),\quad\text{and}\quad\Cbasis{F}_G
\in \pal(2)\ \text{for all}\ G < F.\]

Let \(\calS(\M)\)
be the \(\pal(0)\)-module generated by the elements \(\Cbasis{F}\).

\begin{lemma}\label{lemma:cinS(M)}
    \(\Cbasis{F} \in \calH'_p(\M)\) for all \(F \in \mathcal{L}(\M)\).
\end{lemma}
\begin{lemma}\label{lemma:S(M)is H'_p}
    \(\calS(\M) \cong \calH'_p(\M)\) as \(\pal(0)\)-modules.
\end{lemma}
\begin{lemma}\label{lemma:S(M)is H_p}
    Any Laurent polynomial
    \(f\) can be written as a sum \(\alpha + \beta\) where
    \(\alpha \in \pal(0)\) and \(\beta \in \pal(2)\).
\end{lemma}

\end{document}
A {\it matroid} $\M$ is a combinatorial abstraction of lists of vectors $v_1,v_2,\ldots,v_n$ in a vector space, recording only the information about which subsets of the vectors
are linearly independent or dependent, forgetting their coordinates-- see
Section~\ref{matroid-background-section} for definitions and references.
In groundbreaking work, Adiprasito, Huh and Katz \cite{AHK} affirmed long-standing conjectures of Rota--Heron--Welsh and Mason about vectors and matroids via a new methodology.  Their work employed
a certain graded $\Z$-algebra $A=\bigoplus_{k=0}^r A^k$
called the \emph{Chow ring} for a matroid $\M$ of rank $r+1$, introduced by Feichtner and Yuzvinsky \cite{FY} as a generalization of the Chow ring of DeConcini and Procesi's \emph{wonderful compactifications} for hyperplane arrangement complements \cite{DCP}. 
A remarkable integral Gr\"obner basis result proven by Feichtner and Yuzvinsky \cite[Thm. 2]{FY} shows that each homogeneous component of $A$ is free abelian:  $A^k \cong \Z^{a_k}$ for a positive integer sequence $(a_0,a_1,\ldots,a_r)$.
%, with $a_r=1$ so that there is a {\it degree map} isomorphism $\deg: A^r \overset{\sim}{\longrightarrow} \Z$.

A key step in \cite{AHK} shows not only that 
%the sequence
$(a_0,a_1,\ldots,a_r)$ is
{\it symmetric} and {\it unimodal}, that is,
\begin{align}
\label{symmetry}
&a_k = a_{r-k}\text{ for } k \leq r/2\\
\label{unimodality} 
&a_0 \leq a_1 \leq \cdots \leq a_{\lfloor \frac{r}{2} \rfloor} =
a_{\lceil \frac{r}{2} \rceil} \geq \cdots \geq a_{r-1} \geq a_r,
\end{align}
but in fact proves this as a corollary of something much stronger: the Chow ring $A$ enjoys a trio of properties referred to as the {\it K\"ahler package}, reviewed in Section~\ref{AHK-section} below.  The first of these properties is
{\it Poincar\'e duality}, proving \eqref{symmetry} via a natural
$\Z$-module isomorphism $A^{r-k} \cong \Hom_\Z(A^k,\Z)$.  The second property, called the {\it Hard Lefschetz Theorem}, shows that after tensoring $A$ over $\Z$ with $\R$ to obtain $A_\R=\bigoplus_{k=0}A^k_\R$, one can find {\it Lefschetz elements} $\omega$ in $A^1_\R$ such that multiplication by $\omega^{r-2k}$ gives $\R$-linear isomorphisms $A^k_\R \rightarrow A^{r-k}_\R$ for $k \leq \frac{r}{2}$.  In particular, multiplication by $\omega$ maps $A^k_\R \rightarrow A_\R^{k+1}$ {\it injectively} for $k < \frac{r}{2}$, strengthening the unimodality assertion \eqref{unimodality}. 


Feichtner and Yuzvinsky defined the Chow ring $A(\L_\M,\G)$ for any choice of a {\it building set} $\G$ inside the {\it lattice of flats} $\L_\M$ for the matroid $\M$ and gave their integral Gr\"obner basis presentation in that context; these notions are reviewed in Section~\ref{background-section} below.
%All of these results turn out to hold more generally, for any choice of a {\it building set} $\G$ inside the {\it lattice of flats} $\L_\M$ for the matroid $\M$; these notions are reviewed in Section~\ref{background-section} below.
While the results of \cite{AHK} were proven for the {\it maximal building set} $\G=\L_\M \setminus \{\hat{0}\}$, the Chow ring satisfies the K\"ahler package for any building set (and even for Chow rings of {\it polymatroids}), as shown by Pagaria and Pezzoli \cite[Thm.~4.21]{PagariaPezzoli}.
% However,  Feichtner and Yuzvinsky defined the Chow ring $A(\L_\M,\G)$ for {\it all} building sets and gave their integral Gr\"obner basis presentation in that context. 
% Furthermore, for all building sets $\G$, the ring $A(\L_\M,\G)$ satisfies the K\"ahler package--  this follows for the maximal building set by the results of \cite{AHK}, and for arbitrary building sets (and even for Chow rings of {\it polymatroids}) by work of Pagaria and Pezzoli \cite[Thm.~4.21]{PagariaPezzoli}.

%by applying a result of \cite{AHK}and Ardila, Denham and Huh \cite[Thm.~1.6]{ADH}, showing that the K\"ahler package depends only on the support of the Bergman fan of $\M$, not on its triangulation given by the building set $\G$.






We are interested  in how the {\it Poincar\'e duality} and {\it Hard Lefschetz} properties interact with symmetry.  We consider any subgroup $\subgroup$ of the group $\Aut(\M)$ of symmetries of the matroid $\M$, assuming that the building set $\G$ is also setwise $\subgroup$-stable.  We observe (see Section~\ref{matroid-background-section} below) that in this situation, $\subgroup$ acts via graded $\Z$-algebra automorphisms
on $A(\L_\M, \G)$, giving $\Z \subgroup$-module structures on each $A^k$, and
$\R \subgroup$-module structures on each $A^k_\R$.  One can also check that $A^r \cong \Z$ with trivial $\subgroup$-action, under one additional technical assumption, that $\G$ contains the ground set of
the matroid--see the proof of Corollary~\ref{integral-equivariant-PD-cor} below.  From
this, the Poincar\'e duality pairing immediately gives 
rise to a $\Z \subgroup$-module isomorphism
\begin{equation}
\label{integral-rep-PD-isomorphism}
A^{r-k} \cong \Hom_\Z(A^k,\Z)
\end{equation}
where $g$ in $\subgroup$ acts on $\varphi$ in $\Hom_\Z(A^k ,\Z)$ via $\varphi \mapsto \varphi \circ g^{-1}$; similarly
$A^{r-k} \cong \Hom_\R(A^k,\R)$ as $\R \subgroup$-modules.  
Furthermore, we observe that
%at least for the Chow ring $A(\L_\M,\G_{\max})$ \a{shouldn't this be true for all building sets?} associated with the maximal building set,
one can pick a Lefschetz element $\omega$ 
%as in \cite{AHK} 
which is $\subgroup$-fixed (see Corollary~\ref{AHK-equivariant-Hard-Lefschetz} below), giving $\R \subgroup$-module isomorphisms and injections

\begin{align}
    \label{real-Lefschetz-PD-isomorphism}
    A_\R^{k} &\overset{\sim}{\longrightarrow} A_\R^{r-k} 
     \quad \text{ for }k \leq \frac{r}{2} \nonumber \\
    a &\longmapsto  a \cdot \omega^{r-2k}
\end{align}
\begin{align}
  \label{real-Lefschetz-injection}
        A_\R^{k} &\hookrightarrow A_\R^{k+1} \quad \text{ for }k < \frac{r}{2} \nonumber \\
        a & \longmapsto  a \cdot \omega. 
\end{align}

We wish to view the isomorphism \eqref{real-Lefschetz-PD-isomorphism} and injection \eqref{real-Lefschetz-injection}
as lifting the numerical equality \eqref{symmetry} and inequality \eqref{unimodality} from $\Z$ to the {\it ring of virtual (complex) characters} $R_\C(\subgroup)$.  Recall that this ring $R_\C(G)$ is a subring of the ring of (conjugacy) class functions $\{f: \subgroup \rightarrow \C\}$ with pointwise addition and multiplication.  It is defined as the free $\Z$-submodule with basis given by the irreducible complex characters $\{ \chi_1,\ldots,\chi_M\}$, where the number $M$ of irreducible characters coincides with the number of conjugacy classes of $\subgroup$.  Thus every virtual character $\chi$ in $R_\C(\subgroup)$ has a unique expansion $\chi = \sum_{i=1}^M a_i \chi_i$ for some $a_i \in \Z$.  If $a_i \geq 0$ for $i=1,2,\ldots,M$,
call $\chi$ a {\it genuine character}, and write $\chi \geq_{R_\C(\subgroup)} 0$.
Similarly, write $\chi \geq_{R_\C(\subgroup)} \chi'$ when $\chi-\chi' \geq_{R_\C(\subgroup)} 0$.
There is a surjective ring map
\begin{equation}
\label{characters-to-integers-map}
R_\C(\subgroup) \longrightarrow \Z
\end{equation}
sending a virtual character $\chi$ to its value $\chi(e)$ on the identity $e$ of $\subgroup$, carrying genuine characters $\chi \geq_{R_\C(\subgroup)} 0$ to nonnegative integers $\Z_{\geq 0}$.  Through this map, equalities and inequalities in $R_\C(\subgroup)$ lift inequalities in $\Z$.  For example,
\eqref{real-Lefschetz-PD-isomorphism}, \eqref{real-Lefschetz-injection}
give rise to equalities and inequalities in $R_\C(G)$ of this form
\begin{align}
\label{character-symmetry}
\chi_{A_\R^k}&=\chi_{A_\R^{n-k}} \quad \text{
for }k \leq \frac{r}{2},\\
\label{character-unimodality}
\chi_{A_\R^k} &\leq_{R_\C(G)} \chi_{A_\R^{k+1}}
\quad \text{ for }k < \frac{r}{2},
\end{align}
which lift the equalities and inequalities \eqref{symmetry}, \eqref{unimodality} in $\Z$
 through the map \eqref{characters-to-integers-map}.


Our goal in this paper is to use  Feichtner and Yuzvinsky's Gr\"obner basis result, along with some combinatorics of {\it nested sets} (reviewed in Section~\ref{nested-set-subsection}), to prove a combinatorial strengthening/lifting of the isomorphisms and injections \eqref{integral-rep-PD-isomorphism}, \eqref{real-Lefschetz-PD-isomorphism}, \eqref{real-Lefschetz-injection}.  For the sake of stating this, recall (or see Section~\ref{matroid-background-section} below) that a (simple) matroid $\M$ can be specified by
its {\it lattice of flats} $\L_\M$;  in the case where $\M$ is realized by
a list of vectors $v_1,v_2,\ldots,v_n$ in a vector space, a subset $F \subseteq \{1,2,\ldots,n\}=:E$ is a flat when  $\{v_j\}_{j \in F}$ is linearly closed, meaning that every vector $v_i$ for $i\in E$ that lies in the linear span of $\{v_j\}_{j \in F}$ already has $i$ in $F$.
This $\L_\M$ is a ranked lattice, whose rank function $\rk: \L_\M \rightarrow \{0,1,2\ldots\}$ is modeled after the dimension of the span of $\{v_j\}_{j \in F}$. A building set $\G \subseteq \L_\M$ is a subset of $\L_\M$ satisfying axioms that roughly say every flat $F$ in $\L_\M$ can be ``built" in a certain way from elements of $\G$.
The building set $\G$ distinguishes certain subsets $N=\{F_1,\ldots,F_\ell\} \subset \G$ called {\it $\G$-nested sets}.
To each flat $F$ in the $\G$-nested set $N$, we need a crucial quantity
\begin{equation}
\label{crucial-quantity}
m_N(F):=\rk(F) - \rk(\vee N_{<F})
\end{equation}
where $\vee  N_{<F}$ denotes the lattice join in $\L_\M$ of all elements of $N$ strictly
below $F$.
Then the \emph{Chow ring} $A(\L_\M, \G)$ of $\M$ with respect to the building set $\G$
is presented as a quotient of the
polynomial ring $S:=\Z[x_F]$ having one variable $x_F$ for each flat $F$ in $\G$.  The presentation takes the form 
$
A(\L_\M, \G) := S / (I + J)
$
where $I ,J$ are certain ideals of $S$ defined more precisely in Definition~\ref{Chow-ring-definition} below.
Feichtner and Yuzvinsky exhibited (see Theorem~\ref{FY-GB-theorem}, Corollary~\ref{cor: mon_basis} below) a Gr\"obner basis for $I + J$ that leads to the following standard monomial $\Z$-basis for $A(\L_\M, \G)$,
which we will call the {\it FY-monomials} of $\M$:
$$
\fy:=\left\{x_{F_1}^{m_1}  \cdots x_{F_\ell}^{m_\ell} \colon
N:=\{F_1, \cdots,  F_\ell\} \text{ is } \G\text{-nested, and } 
0 \leq m_i < m_N(F_i) \text{ for }i=1,2,\ldots,\ell.\right\}
$$


The subset $\FY^k$ of FY-monomials $x_{F_1}^{m_1} \cdots x_{F_\ell}^{m_\ell}$ of total degree $m_1+\cdots+m_\ell=k$ then gives a $\Z$-basis for $A^k$.
One can readily check
(see Corollary~\ref{Chow-ring-carries-perm-reps-cor}) that the group $\subgroup$ permutes the $\Z$-basis $\FY^k$ for $A^k$, 
endowing $A^k$ with the structure of
a {\it permutation representation}, or {\it $\subgroup$-set}.  
Our main result, proven in Section~\ref{main-theorem-section}, is this strengthening of the isomorphisms and injections \eqref{integral-rep-PD-isomorphism}, \eqref{real-Lefschetz-PD-isomorphism}, \eqref{real-Lefschetz-injection}.


\begin{thm}
\label{main-theorem}
Let $\M$ be a simple matroid rank $r+1$ on ground set $E$.  Let $\subgroup$
be a group automorphisms of $\M$, and $\G$ a building set in
$\L_\M$
that contains $E$, is setwise $G$-stable, and satisfies this
{\it stabilizer condition}\footnote{This condition was missing in {\tt arXiv} version 2 of this paper--  the authors 
thank R. Pagaria for pointing out the issue.}:
\begin{equation}
\label{eq:stabilizer-condition}
  \text{for any }\G\text{-nested set }
   N=\{F_i\}_{i=1,\ldots,\ell}, \text{ if }g \in G\text{ has } 
    g(N)=N, \text{ then }g(F_i)=F_i \text{ for } i=1,\ldots,\ell.
\end{equation}
Then there exist
\begin{itemize}
\item[(i)]
$\subgroup$-equivariant  bijections 
$
\pi: \FY^k  \overset{\sim}{\longrightarrow}  \FY^{r-k}
$ for $k \leq \frac{r}{2}$, and
\item[(ii)]
$\subgroup$-equivariant injections
$
\lambda: \FY^k  \hookrightarrow  \FY^{k+1}
$
for $k < \frac{r}{2}$.
\end{itemize}
\end{thm}


\begin{example} 
        Let $\M=U_{4,5}$ be the uniform matroid of rank $4$ on $E=\{1,2,3,4,5\}$, associated to $5$ {\it generic} vectors $v_1,v_2,v_3,v_4,v_5$ in a $4$-space, so that any quadruple $v_i,v_j,v_k,v_\ell$ is linearly independent.  Choose $\G=\G_{\max}=\L_\M\setminus\{\varnothing\}$, the \emph{maximal} building set.
        Then $\L_\M$ has these flats of various ranks:

\begin{center}
\begin{tabular}{|c|c|}\hline
rank & flats $F \in \L_\M$ \\ \hline\hline
 $0$ & $\varnothing$ \\ \hline
 $1$ & $1,2,3,4,5$ \\ \hline
 $2$ & $12,13,14,15,23,24,25,34,35,45$ \\ \hline
 $3$ & $123,124,125,134,135,145,234,235,245,345$ \\ \hline
 $4$ & $E=12345$ \\ \hline
\end{tabular}
\end{center}
The Chow ring $A(\L_\M, \G)=S/(I+J)$, where 
$
S= \Z[ x_i, x_{jk}, x_{\ell m n}, x_E]
$
with
$\{i\}, \{j,k\}, \{\ell, m, n\}$ running through all one, two and three-element subsets of 
$E=\{1,2,3,4,5\}$,
and 
$$
I=\Big( x_F x_{F'} \Big)_{F \not\subset F', F' \not \subset F},
\qquad
J=\bigg( x_i 
+ \sum_{\substack{1 \leq j < k \leq 5\\i \in \{j,k\}}} x_{jk}
+ \sum_{\substack{1 \leq \ell  < m < n \leq 5\\i \in \{\ell,m,n\} }} x_{\ell m n}
\,\,\, + x_E \bigg)_{i=1,2,3,4,5}.
$$
The FY-monomial bases for $A^0,A^1,A^2,A^3$ are shown here,
together with the $\subgroup$-equivariant maps $\lambda$:
$$
\begin{array}{cccccccc}
\bf{FY^0} & & \bf{FY^1}& & \bf{FY^2} & & &\bf{FY^3} \\
 & & & & & & & \\
1 &\overset{\lambda}{\longmapsto}& x_E &\overset{\lambda}{\longmapsto} & x_E^2 &  & &x_E^3 \\
 & & & & & & &\\
  & & x_{ijk} &\overset{\lambda}{\longmapsto} & x_{ijk}^2& & &\\
      & &  1 \leq i<j<k \leq 5 & & & & & \\
       & & & & & & &\\
  & &x_{ij} &\overset{\lambda}{\longmapsto} &x_{ij} \cdot x_E& & &\\
   & & 1 \leq i<j \leq 5& & & &  &\\
\end{array} 
$$
Therefore in this case, the ranks of the free $\Z$-modules $(A^0,A^1,A^2,A^3)$ form the symmetric, unimodal sequence $(a_0,a_1,a_2,a_3)=(1,21,21,1)$. 
Here the bijection $\pi: \fy^0 \rightarrow \fy^3$ 
necessarily maps $1 \longmapsto x_E^3$, and 
the bijection $\pi: \fy^1 \rightarrow \fy^2$  coincides with
the map $\lambda: \fy^1 \rightarrow \fy^2$ above.  
\end{example}

We wish to also view Theorem~\ref{main-theorem} as lifting \eqref{character-symmetry}, \eqref{character-unimodality} from the virtual character ring $R_\C(G)$ to the 
{\it Burnside ring $B(\subgroup)$ of virtual $G$-sets}.  Recall that to define the Burnside ring $B(G)$
(see, e.g., Bouc \cite{Bouc}), one starts with a free $\Z$-module having as basis the $\subgroup$-equivariant isomorphism classes $[X]$ of finite $\subgroup$-sets $X$.  Then $B(\subgroup)$ is the quotient $\Z$-module that mods out by the span of all elements $[X \sqcup Y] - ([X]+[Y])$.  
Multiplication in $B(\subgroup)$ is induced from the rule $[X] \cdot [Y] = [X \times Y]$.
It turns out that $B(\subgroup)$ has a $\Z$-basis given by the isomorphism classes of $\{ [\subgroup/\subgroup_i] \}_{i=1}^N$ of the transitive $G$-sets $G/G_i$ as $\subgroup_1,\ldots,\subgroup_N$ run through representatives of the $\subgroup$-conjugacy classes of subgroups of $\subgroup$. 
Thus every element $b$ of $B(\subgroup)$ has a unique expansion $b = \sum_{i=1}^N a_i [\subgroup/\subgroup_i]$ for some $a_i \in \Z$.  If $a_i \geq 0$ for $i=1,2,\ldots,N$, then write $b \geq_{B(\subgroup)} 0$ and call $b$ a {\it genuine} element of $B(G)$, 
since it is the class $b=[X]$ 
of the (genuine, not virtual) 
$G$-set $X$ that has a total of  $\sum_{i=1}^N a_i$ disjoint $G$-orbits, among which $a_i$ of the orbits are isomorphic to the transitive $G$-set $G/G_i$.
Similarly, write $b \geq_{B(\subgroup)} b'$ when $b-b' \geq_{B(\subgroup)} 0$.  The map which sends the class $[X]$ of a $\subgroup$-set $X$ to the character $\chi_{\C[X]}(g)=\#\{x \in X:g(x)=x\}$ of its $\subgroup$-permutation representation $\C[X]$ gives a natural ring map
\begin{equation}
    \label{Burnside-to-character-map}
B(\subgroup) \longrightarrow R_\C(\subgroup)
\end{equation} 
This map sends genuine elements $b \geq_{B(\subgroup)} 0$ in $B(\subgroup)$ to genuine characters $\chi \geq_{R_\C(\subgroup)} 0$ in $R_\C(G)$.  In this way, equalities and inequalities in $B(\subgroup)$ lift equalities and inequalities in $R_\C(\subgroup)$.
For example, Theorem~\ref{main-theorem} (i),(ii) 
lead to equalities and inequalities in $B(G)$ of this form
\begin{align}
\label{Burnside-symmetry}
[\FY^k]&=[\FY^{n-k}] 
\quad \text{ for }k \leq \frac{r}{2},\\
\label{Burnside-unimodality}
[\FY^k] &\leq_{B(G)} [\FY^{k+1}]
\quad \text{ for }k < \frac{r}{2},
\end{align}
which lift the equalities and inequalities \eqref{character-symmetry}, \eqref{character-unimodality} in $R_\C(G)$
 through the map \eqref{Burnside-to-character-map}.


Before proving Theorem~\ref{main-theorem} in Section~\ref{main-theorem-section}, the background Section~\ref{background-section} reviews properties of building sets of atomic lattices, Chow rings, and matroids. It also collects a few simple observations,
including one new and crucial numerical fact about $\G$-nested sets, Lemma~\ref{lem:crucial-numerical-fact}.
Section~\ref{conjectures-section} pursues the theme of lifting inequalities in $\Z$ to inequalities $R_\C(G)$ and in $B(G)$, going beyond unimodality to conjectures that would extend recent log-concavity and total positivity conjectures for Hilbert functions of Chow rings $A(\L_\M, \G)$.
Section~\ref{sec: further-questions} poses additional questions
and conjectures.



\section*{Acknowledgements}
The authors thank Federico Ardila, Alessio D'Ali, Graham Denham, Chris Eur, Eva-Maria Feichtner, Luis Ferroni, June Huh, Matt Larson, Hsin-Chieh Liao,  Diane Maclagan, Matt Maestroni, Jacob Matherne, Ethan Partida, Connor Simpson, Lorenzo Vecchi and Peter Webb for helpful conversations and references.  They particularly thank Roberto Pagaria for
alerting them to an issue in the previous statement of Theorem~\ref{main-theorem} that omitted the stabilizer condition \eqref{eq:stabilizer-condition}. 
They thank an anonymous referee for helpful suggestions that improved the exposition.  Additionally, they are grateful to Trevor Karn for his wonderful {\tt Sage/Cocalc} code that checks whether a symmetric group representation is a permutation representation.  The authors received partial support from NSF grants 
DMS-1949896, DMS-1745638, DMS-2053288, DMS-2230648. The second author received partial support from PROM project no. POWR.03.03.00-00-PN13/18.


%First author received partial support from NSF grant DMS-1949896. First and second authors received partial support from NSF RTG grant DMS-1745638. Second author received partial support from PROM project no.~POWR.03.03.00-00-PN13/18. Third author received partial support from NSF grant DMS-2053288.


